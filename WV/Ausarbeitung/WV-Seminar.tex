\documentclass[12pt,twoside]{article}
%%%%%%%%%%%%%%%%%%%%%%%%%%%%%%%%%%%%%%%%%%%%%%%%%%%%%%%%%%%%%
% Meta informations:
\newcommand{\trauthor}{Jan Fabian Schmid}
\newcommand{\trtype}{Seminar Paper} %{Seminararbeit} %{Proseminararbeit}
\newcommand{\trcourse}{Knowledge Processing}
\newcommand{\trtitle}{Examination of a framework for multi-objective analysis of computational models}
\newcommand{\trmatrikelnummer}{6440383}
\newcommand{\tremail}{2schmid@informatik.uni-hamburg.de}
\newcommand{\trarbeitsbereich}{Knowledge Technology, WTM}
\newcommand{\trdate}{10.11.2015}

%%%%%%%%%%%%%%%%%%%%%%%%%%%%%%%%%%%%%%%%%%%%%%%%%%%%%%%%%%%%%
% Languages:

% Falls die Ausarbeitung in Deutsch erfolgt:
% \usepackage[german]{babel}
% \usepackage[T1]{fontenc}
% \usepackage[latin1]{inputenc}
% \usepackage[latin9]{inputenc}	 				
% \selectlanguage{german}

% If the thesis is written in English:
\usepackage[english]{babel} 						
\selectlanguage{english}

%%%%%%%%%%%%%%%%%%%%%%%%%%%%%%%%%%%%%%%%%%%%%%%%%%%%%%%%%%%%%
% Bind packages:
\usepackage{acronym}                    % Acronyms
\usepackage{algorithmic}								% Algorithms and Pseudocode
\usepackage{algorithm}									% Algorithms and Pseudocode
\usepackage{amsfonts}                   % AMS Math Packet (Fonts)
\usepackage{amsmath}                    % AMS Math Packet
\usepackage{amssymb}                    % Additional mathematical symbols
\usepackage{amsthm}
\usepackage{booktabs}                   % Nicer tables
%\usepackage[font=small,labelfont=bf]{caption} % Numbered captions for figures
\usepackage{color}                      % Enables defining of colors via \definecolor
\definecolor{uhhRed}{RGB}{254,0,0}		  % Official Uni Hamburg Red
\definecolor{uhhGrey}{RGB}{122,122,120} % Official Uni Hamburg Grey
\usepackage{fancybox}                   % Gleichungen einrahmen
\usepackage{fancyhdr}										% Packet for nicer headers
%\usepackage{fancyheadings}             % Nicer numbering of headlines

%\usepackage[outer=3.35cm]{geometry} 	  % Type area (size, margins...) !!!Release version
%\usepackage[outer=2.5cm]{geometry} 		% Type area (size, margins...) !!!Print version
%\usepackage{geometry} 									% Type area (size, margins...) !!!Proofread version
\usepackage[outer=3.15cm]{geometry} 	  % Type area (size, margins...) !!!Draft version
\geometry{a4paper,body={5.8in,9in}}

\usepackage{graphicx}                   % Inclusion of graphics
%\usepackage{latexsym}                  % Special symbols
\usepackage{longtable}									% Allow tables over several parges
\usepackage{listings}                   % Nicer source code listings
\usepackage{multicol}										% Content of a table over several columns
\usepackage{multirow}										% Content of a table over several rows
\usepackage{rotating}										% Alows to rotate text and objects
\usepackage[hang]{subfigure}            % Allows to use multiple (partial) figures in a fig
%\usepackage[font=footnotesize,labelfont=rm]{subfig}	% Pictures in a floating environment
\usepackage{tabularx}										% Tables with fixed width but variable rows
\usepackage{url,xspace,boxedminipage}   % Accurate display of URLs

%%%%%%%%%%%%%%%%%%%%%%%%%%%%%%%%%%%%%%%%%%%%%%%%%%%%%%%%%%%%%
% Configurationen:

\hyphenation{whe-ther} 									% Manually use: "\-" in a word: Staats\-ver\-trag

%\lstloadlanguages{C}                   % Set the default language for listings
\DeclareGraphicsExtensions{.pdf,.svg,.jpg,.png,.eps} % first try pdf, then eps, png and jpg
\graphicspath{{./src/}} 								% Path to a folder where all pictures are located
\pagestyle{fancy} 											% Use nicer header and footer

% Redefine the environments for floating objects:
\setcounter{topnumber}{3}
\setcounter{bottomnumber}{2}
\setcounter{totalnumber}{4}
\renewcommand{\topfraction}{0.9} 			  %Standard: 0.7
\renewcommand{\bottomfraction}{0.5}		  %Standard: 0.3
\renewcommand{\textfraction}{0.1}		  	%Standard: 0.2
\renewcommand{\floatpagefraction}{0.8} 	%Standard: 0.5

% Tables with a nicer padding:
\renewcommand{\arraystretch}{1.2}

%%%%%%%%%%%%%%%%%%%%%%%%%%%%
% Additional 'theorem' and 'definition' blocks:
\theoremstyle{plain}
\newtheorem{theorem}{Theorem}[section]
%\newtheorem{theorem}{Satz}[section]		% Wenn in Deutsch geschrieben wird.
\newtheorem{axiom}{Axiom}[section] 	
%\newtheorem{axiom}{Fakt}[chapter]			% Wenn in Deutsch geschrieben wird.
%Usage:%\begin{axiom}[optional description]%Main part%\end{fakt}

\theoremstyle{definition}
\newtheorem{definition}{Definition}[section]

%Additional types of axioms:
\newtheorem{lemma}[axiom]{Lemma}
\newtheorem{observation}[axiom]{Observation}

%Additional types of definitions:
\theoremstyle{remark}
%\newtheorem{remark}[definition]{Bemerkung} % Wenn in Deutsch geschrieben wird.
\newtheorem{remark}[definition]{Remark} 

%%%%%%%%%%%%%%%%%%%%%%%%%%%%
% Provides TODOs within the margin:
\newcommand{\TODO}[1]{\marginpar{\emph{\small{{\bf TODO: } #1}}}}

%%%%%%%%%%%%%%%%%%%%%%%%%%%%
% Abbreviations and mathematical symbols
\newcommand{\modd}{\text{ mod }}
\newcommand{\RS}{\mathbb{R}}
\newcommand{\NS}{\mathbb{N}}
\newcommand{\ZS}{\mathbb{Z}}
\newcommand{\dnormal}{\mathit{N}}
\newcommand{\duniform}{\mathit{U}}

\newcommand{\erdos}{Erd\H{o}s}
\newcommand{\renyi}{-R\'{e}nyi}
%%%%%%%%%%%%%%%%%%%%%%%%%%%%%%%%%%%%%%%%%%%%%%%%%%%%%%%%%%%%%
% Document:
\begin{document}
\renewcommand{\headheight}{14.5pt}

\fancyhead{}
\fancyhead[LE]{ \slshape \trauthor}
\fancyhead[LO]{}
\fancyhead[RE]{}
\fancyhead[RO]{ \slshape \trtitle}

%%%%%%%%%%%%%%%%%%%%%%%%%%%%
% Cover Header:
\begin{titlepage}
	\begin{flushleft}
		Universit\"at Hamburg\\
		Department Informatik\\
		\trarbeitsbereich\\
	\end{flushleft}
	\vspace{3.5cm}
	\begin{center}
		\huge \trtitle\\
	\end{center}
	\vspace{3.5cm}
	\begin{center}
		\normalsize\trtype\\
		[0.2cm]
		\Large\trcourse\\
		[1.5cm]
		\Large \trauthor\\
		[0.2cm]
		\normalsize Matr.Nr. \trmatrikelnummer\\
		[0.2cm]
		\normalsize\tremail\\
		[1.5cm]
		\Large \trdate
	\end{center}
	\vfill
\end{titlepage}

	%backsite of cover sheet is empty!
\thispagestyle{empty}
\hspace{1cm}
\newpage

%%%%%%%%%%%%%%%%%%%%%%%%%%%%
% Abstract:

% Abstract gives a brief summary of the main points of a paper:
\section*{Abstract}
  Your text here...

% Lists:
\setcounter{tocdepth}{2} 					% depth of the table of contents (for Seminars 2 is recommented)
\tableofcontents
\pagenumbering{arabic}
\clearpage

%%%%%%%%%%%%%%%%%%%%%%%%%%%%
% Content:

% the actual content, usually separated over a number of sections
% each section is assigned a label, in order to be able to put a
% crossreference to it

\section{Introduction}
\label{sec:introduction}

\subsection{Problem}
The studied paper \cite{doncieux2015multi} tries to cover the problem of application of complex computational models.
In more and more areas computational models are used for increasingly hard scientific issues.
Doncieux et al. break the problem down to the epistemic opacity. Which they define as
\begin{align*}
	&\text{\textit{A process is epistemically opague relative to a cognitive agent X at time t just in}}\\
	&\text{\textit{case X does not know at t all of the epistemically relevant elements of the process.}}
\end{align*}
Epistemic opacity may stem from a complicated mathematical foundation to the model, from the complex interaction of different model parts with each other and from the effect of different input-parameters on the model.
They say: '[...], the computational model is in itself a compöex system top study in order to unravel its unforseen features' \cite{doncieux2015multi} p. 3.
This problem can lead to an extensive search for the 'best' values for the model in parameter space.
The researcher might have an intuition for some good parameter values, but these are biased by his assumptions about the model. 

Furthermore in a complex system it might not be obvious what properties of the model are actually desireable. Hence the 'best' model parameters wouldn't be recognized if found.
Since the model should lead to novel knowledge and proven or disproved assumptions about the modelled system interesting parameter-sets are desired. Interesting parameters should be somehow different from other parameters and state something about the model.

To overcome the epistemic opacity the researcher would have to analyse the complex model by approximating it with a model again. This step might be necessary multiple times, each step using a more abstract and simple model, until it is easy enough to understand by the researcher. The knowledge gained from the simpler models could then be used to understand the more complex models. At some point the problem is undestood well enough to use the primary model with confident insight.

\subsection{Suggested solution}
Because modeling a model becomes exponentially intricate and success isn't guaranteed it can hardly be a feasible approach to study a problem. The paper suggests to exploit the fact, that the computational model can be computed in a 'huge numer of experiments' \cite{doncieux2015multi} p. 3.
They suggest a framework method that once set up searches automatically for interesting model parameters, which can then be analysed.
The framework requires that a set of functions with three properties:
\begin{enumerate}
	\item The functions measure the model performance.
	\item For optimal model performance the functions have to be maximized or minimized.
	\item Some functions are contrary to each other, as optimizing one function value worsens an other.
\end{enumerate}
The given function set maps parameter space to behaviour space, calculating many possible solutions allows to find relations between both spaces.
Since the performance of individual functions is contrary to each other multi-objective algorithms can be used to find sets of trade-off solutions \cite{doncieux2015multi} p. 3.
Doncieux et al. then suggest to use evolutionary multi-objective algorithms, since they are efficient and versatile in usage.
Following the discovery of interesting parameter sets, the paper suggests some tools to acquire knowledge about the relations of parameter values and model behaviours.

\subsection{Goals and structure of this paper}
This paper is supposed to give a in-depth analysis of the presented paper \cite{doncieux2015multi}.
Firstly in Chapter \ref{sec:background} some background, necessary to understand the main concepts of the paper, will be given.
The proposed frame work of Doncieux et al. will be presented in a summarized form in Chapter \ref{sec:model}.
In the following chapter \ref{sec:analysis} the model and its results will be critically reviewed.
Especially the applicability of and assumptions made by the framework will be analysed.
Chapter \ref{sec:concl} will conclude the paper by stating the achieved insights on the usage and applicability of the framework. 
%- Is this framework applicable to all kinds of computational methods? What conditions must be met by the computational method?
\section{Background Information}
\label{sec:background}
The crucial concepts to understand the framework for computational methods are:

Multi-objective problems and what a optimal solution to such an ambiguous problem looks like. These topics will be discussed in section \ref{back:multi-opt}.
 
The functionality of evolutionary algorithms (\ref{back:evo}).

Section \ref{back:evo_in_multi-opt} brings these concepts in context for the usage of evolutionary algorithms in Multi-objective-optimiziation.

And finally in section \ref{back:indicators} some indicators are introduced, that will be used to analyse the results of the multi-objective-optimization.

\subsection{Multi-objective optimization}
\label{back:multi-opt}
%\cite{miettinen2012nonlinear}\\
Doncieux et al. \cite{doncieux2015multi} formalize optimization problems as

\begin{align*}
	\text{Find the parameter }\textbf{X} &=
	\left \{
	\begin{tabular}{c}
		$x_1$\\
		$x_2$\\
		\vdots\\
		$x_m$
	\end{tabular}
	\right \}
	\in \mathbb{R}^m \text{that maximizes (or minimizes) f(\textbf{X}) under}\\
		 \text{the constraints:}\\
	g_j(\textbf{X}) &\le 0 \text{,  } j = 1,2,... ,p\\
	l_j(\textbf{X}) &= 0 \text{,  } j = 1,2,... ,q
\end{align*}

For $m = 1$ such a task is called mono-objective problem, whereas it's a multi-objective problem for any $m > 1$.

As soon as you want to examine a real world like problem it's very likely that the solution won't be a single one dimensional value. Instead there will be multiple objectives to be considered simultanesously.
Fonseca and Fleming \cite{fonseca1995overview} state that an optimal solution to a multi dimensional problem is commonly not optimal for each of the objectives.
This should be the case as long as the different objectives are somehow conflicting.
They say further that you would rather aim at a solution that is at least acceptable for all sub-objectives.
This however isn't a explicit problem statement. In a non-trivial case there will be several quite different alternative approaches to the problem that are optimal in some aspects.
All these approaches should be valid solutions given by a multi-objective optimization-algorithm.
To allow this to happen the Pareto dominance relation and the Pareto-optimal set are introduced.
A solution $\mathbf{X_1}$ dominates another solution $\mathbf{X_2}$ if $\mathbf{X_1}$ is not worse than $\mathbf{X_2}$ for any objective and for at least one objective strictly better. \cite{doncieux2015multi} 
A solution is called Pareto-optimal if it isn't dominated. All Pareto-optimal solutions form the Pareto-optimal set. 
Since the Pareto dominance relation is a partial ordering it allows any number of optimal solutions.
The Pareto-optimal set is just the set of interesting solutions, because each of these solutions is the best in some aspects.

\subsection{Evolutionary algorithms}
\label{back:evo}
\cite{eiben2004introduction}\\

\subsection{Evolutionary Algorithms for Multi-objective optimization}
\label{back:evo_in_multi-opt}
As common optimization-algorithms like gradient descent or simulated annealing aren't suited to deal with multi-objective problems \cite{fonseca1995overview} other approaches have been used.
One of the strenghts evolutionary algorithms have, is to pursue different possible solutions at the same time. That's because in a well structured evolutionary algorithm the surviving individuals aren't only associated with the single best solution at the given time, but also momentary suboptimal solutions are beeing tracked.
Through crossover partially optimal parts of solutions can be exchanged as a whole. 
Therefore the Pareto front, the current Pareto-optimal set during execution of the algorithm, can be explored effectively.

As evolutionary algorithms impose no mathematical constraints on the examined problem. The stochastical approach of evolutionary algorithms however entails in application to calculate many possible solutions to work properly. Also evolutionary algorithm doesn't ensure to find global optima, instead it converges to some local optimum. Accordingly only an approximation to the Pareto optimal solutions will be found.

The concept used by Doncieux et al. to gain knowledge about a system by analysing the results of a multi-objective optimization algorithm is called INNOVIZATION as they state \cite{doncieux2015multi} (p. 5) this stands for 'INNOVation through optimIZATION'.
Evolutionary multi-objective optimization algorithms were already used succesfully with this approach as shown in \cite{efstratiadis2010one}, where Efstratiadis and Koutsoyiannis sum up the experiance with the INNOVIZTION method for calibration of hydrological models.

\subsection{Indicators for multi-objective analysis}
\label{back:indicators}
To be able to analyse the gained information from multi-objective-optimization on the examined problem Doncieux et al. \cite{doncieux2015multi} (p. 7) introduce some indicators.
In the following it is without loss of generality assumed, that all objective-functions are to be maximized to be optimal.
The evolutionary multi-objective-optimization is used multiple times on the problem. Each run generates a set of approximations for the Pareto optimal solutions called $\chi_i$ for the $i$-th run.
The required indicators are defined as follows.
\begin{itemize}
	\item The attainment function $\alpha_\chi(z)$ returns the probability to find in the set of results $\chi$ \textbf{at least one} solution $x$ that dominates $z$. This function can't be computed directly, but it can be approximated with the empirical attainment function $\alpha_r(z)$ over $r$ sets of approximation results as mean number of sets in which a solution is found that dominates $z$.
	\item The attainment surface $\Psi_p$ to a attainment function value $p$ is defined as the hyper surface in behaviour space over points with probability $p$ given by the attainment function.
	Therefore the 0-attainment surface $\Psi_0$ covers the set of non-dominated solutions and the 1-attainment surface $\Psi_0$ represents the worst performing solutions, that will certainly be dominated.
	\item The hypervolume indicator $I^p_H(\chi)$ is defined for a particular reference point $p$ for a set of results $\chi_0$. A simplified attainment function $\beta\chi(z)$ definition is used for the hypervolume indicator, it returns $1$ if there is a solution in $\chi$, that dominates z, and $0$ otherwise. The hypervolume indicator is then defined as:
	\begin{align*}
		I^p_H(\chi_0) &= \int_{\psi_p}\beta\chi(z) dz
	\end{align*}
	With $\psi_p$ containing all points, that are equal or greater than $p$ in every dimension.
	Therefore $I^p_H(\chi)$ measures the area of points above $p$ that are dominated by a solution in $\chi$.
	\item $\eta(\chi)$ is the point with the least possible value for each objective found in $\chi$. So for each dimension all solutions in $\chi$ are scanned for the minimum value.
	When $\eta^i(\chi)$ is the nadir point to the $i$-th set of solutions, the conservative nadir point $\bar{\eta}$ is defined over $r$ sets of solutions and its values are the maximum value for each dimension found in any $\eta^i(\chi)$ with $i\in r$.
\end{itemize} 

\section{Framework description and application}
\label{sec:model}
The proposed framework by Doncieux et al. \cite{doncieux2015multi} aims at automatically finding parameter values worth to be studied further, that can then be analysed by experts.
There are three steps to application of the framework.
\begin{enumerate}
	\item The definition of functions to evaluate the performance of the examined computational method.
	\item Obtain an approximation of the Pareto-optimal solutions by application of evolutionary multi-objective optimization solutions.
	\item Knowledge creation by analysing the solutions.
\end{enumerate}
These steps are specified more in detail in the following sections.

\subsection{Seach space and objective functions}
First the search space as part of the parameter space must be stated. For a set of m unbound parameters it would be $\mathbb{R}^m$. 

The framework relies on the comparison of different Pareto optimal solutions to find interesting parameter values, therefore multiple contrary objective functions have to be defined.
They need to be contrary in terms of not linearly dependent, since such a dependency would lead to a set of Pareto optimal solutions with only one member that is optimal in each dimension.
A simple second objective for an computational model whose performance actually only depends on one variable might be found in the computational cost.
With increasing computaion time the algorithm might find better and better solutions as more time is given. The optimal parameters would then be a compromise between quality of solution and time consumption.

\subsection{Application of multi-objective-optimization}
If ideally no promising parameter vectors shall be missed it is important to make a dense search in parameter space, otherwise behaviors that only accure in a small area in parameter space might be skipped.
As explained in chapter \ref{back:evo_in_multi-opt} evolutionary algorithms are a good choice for the multi-objective-optimization.
As the results from evolutionary algorithms strongly depend on their parametrisation, it is important to verify the results somehow.
Therefore the variability of results to successive runs of evolutionary multi-objective-optimization can be checked.

\subsection{Analysis of results}

The framework\\
- How does it use Multi-objective optimization and evolutionary algorithms?\\
- Why does it solve the problem?\\
- What are the steps of application?\\

\section{Framework analysis}
\label{sec:analysis}

Results of the presented experiments\\
- Were the results expected?\\
- Positive or negative outcomes?\\
Usability of the model\\
- Effort and benefits of using it\\
Applicability (research question)\\
- What assumptions and constraints are made on the computational model?\\
- What are fields of application for the framework? (Is it everywhere applicable?)\\

\section{Conclusion}
\label{sec:concl}

How reasonable is the approach of the paper?\\
Where is it applicable?

%%%%%%%%%%%%%%%%%%%%%%%%%%%%%%%%%%%%%%
% hier werden - zum Ende des Textes - die bibliographischen Referenzen
% eingebunden
%
% Insbesondere stehen die eigentlichen Informationen in der Datei
% ``bib.bib''
%
\newpage
\bibliographystyle{plain}
\addcontentsline{toc}{chapter}{Bibliography}% Add to the TOC
\bibliography{bib}

\end{document}


