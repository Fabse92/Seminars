\documentclass[11pt]{scrartcl}
\usepackage[latin1]{inputenc}
\usepackage[german]{babel}
\usepackage[T1]{fontenc}
\usepackage{latexsym}
\usepackage{stmaryrd}
\usepackage{amsmath}
\usepackage{amssymb}
\usepackage{amsxtra}
\usepackage[round]{natbib}

\newtheorem{theorem}{Theorem} 
\newtheorem{definition}[theorem]{Definition} 


\begin{document}

\title{  �ber die Umf�rbbarkeit\\ roter Gummib�ren } 
\subtitle{Eine informelle Einf�hrung in die Iterationstheorie}
\author{Michael K\"ohler-Bu{\ss}meier}
\date{\small Hausarbeit im Modul FGI-3, WS 2042/2043\\
  Fachbereich Informatik \\ Universit�t Hamburg\\[4mm]
  \today }

\maketitle


\bigskip
\begin{abstract}
  \small\noindent\textbf{Zusammenfassung:}

  Rote Gummib�ren k�nnen die Welt retten!
\end{abstract}

\tableofcontents
\newpage

\section{Einleitung}

Warum brach die 40-j�hrige Hegemonie der Sowjetunion in Mittel- und
Osteuropa im Jahre 1989 innerhalb von wenigen Monaten zusammen? 
Dies liegt an den roten Gummib�ren!


\subsection{Motivation}

Warum sollte man sich mit der Theorie roter Gummib�ren besch�ftigen?


\subsection{Probleme und  Fragen}

Will man sich mit roten Gummib�ren besch�ftigen, dann treten folgende
Probleme und Fragen auf:


\subsection{Die Theorie der Iteration von Ensembles}

Der Ansatz der hier betrachtet werden soll ist die Theorie der
Iteration von Ensembles. Dabie handelt es sich um.....


\subsection{Aufbau der Arbeit}

Die Arbeit hat den folgenden Aufbau: Um zu einer Theorie der roten
Gummib�ren zu gelangen, besch�ftigen wir uns in
Kapitel~\ref{sec:Iterierte-Umf�rbungen} mit ....



\section{Iterierte Umf�rbungen }
\label{sec:Iterierte-Umf�rbungen}

Im folgenden betrachten wir Umf�rbungen von T�ten sowie deren
Iteration.  Hierbei ist insbesondere der Grenzwertprozess von
Interesse.



\subsection{  Iteration, Stabilisation }

Wir nehmen eine vorgegebene Mengen an Farben $C$ an.  Der Einfachheit
halber identifizieren wir eine T�te $T$ mit $n$ Gummib�ren mit dem
Intervall $[1, \ldots, n]$.


\begin{definition}[F�rbung]
  Eine \emph{F�rbung} ist eine Abbildung $f: [1, \ldots, n] \to C$.

  Sei $F$  die Menge aller F�rbungen
\end{definition}

Ein \emph{Funktional} ist eine Funktion, die Funktionen als Argumente
hat, d.h. .....

\begin{definition}[Umf�rbung]
  Eine \emph{Umf�rbung} ist eine Funktional $u: F \to F$.
\end{definition}




\begin{definition}    
  Sei die F�rbung $f: [1, \ldots, n] \to C$ gegeben.
  Die Iteration einer Umf�rbung $u: F \to F$ ist
  \[
  \begin{array}{rcl}
    u^0(f) &:=& f \\
    u^{n+1}(f) &:=&     u(u^{n}(f)) 
  \end{array}
  \]
\end{definition}

Wir h�tten es gerne, dass sich $u^{n}(f)$ f�r $n \to \infty$
stabilisiert.




\subsection{  Ordnungen }

In \citep{hans-riegel-1994} findet sich der folgende Satz:

\begin{theorem}[Hans und Riegel, 1994]
  Zu jeder wohlgeordneten Menge von Gummib�renfarben ...
\end{theorem}

Historisch betrachtet findet sich der Wohklordnungsbegriff aber
bereits schon \citep{riegel-1993} angelegt.

\subsection{  Eindeutigkeit  }

Es gibt eine Besonderheit des Wohlordnungssatz auf Gummib�renfarben:
Der Wohlordnungssatz auf Gummib�renfarben garantiert die
Stabiliserung.  Er garantiert aber nicht die Eindeutigkeit des
Endergebnisses.  Das Endergebnis h�ngt von der Auswahlfunktion $g:
\mathbb{N} \to [1, \ldots, k]$ auf dem Umf�rbungsensemble ab.  Es
ergibt sich also sofort die Frage: F�r welche Umf�rbungsensembles ist
auch das Endergebnis eindeutig?



\section{Verwandte Arbeiten und Ans�tze}

Wir finden in der Literatur eine Reihe �hnlicher Ans�tze, von denen
wir einige vorstellen wollen.

\paragraph{Ondulierten Umf�rbung}
\paragraph{Iterierte Verf�rbung}
\paragraph{Gef�rbte Iteration}





\section{Ausblick und Zusammenfassung}

\subsection{ Zusammenfassung}

\subsection{Ausblick}

In dieser Hausarbeit habe ich einiges nur kurz angerissen bzw. ganz
weggelassen, weil es den Rahmen des Seminars sprengt.

Insbesondere habe ich nicht die Theorie der Umf�rbung auf unendlich
gro�en T�ten behandelt.  Diese Theorie basiert prinzipiell auch auf
den hier behandelten Konzepten, wobei daruaf zu achten ist, dass....



\subsection{Bezug zum M.Sc. Studium}

Abschlie�end m�chte ich die Relevanz des Themas f�r das weitere
Studium im Master skizzieren....



\newpage\bibliographystyle{dinat}
\bibliography{mybib}

\end{document}

